\documentclass[a4paper,12pt]{article}
\usepackage[utf8]{inputenc}
\usepackage[T1]{fontenc}
\usepackage[brazilian]{babel}
\usepackage{indentfirst}
\usepackage{abntex2cite}
%\usepackage{hyperref}
\usepackage{graphicx}
\usepackage{amsmath}
\usepackage{setspace}
\usepackage{times}

\title{Título do Trabalho}
\author{João Vínicius\\
Victor Jorge Carvalho Chaves}
\date{11/07/2024}

\begin{document}

\maketitle
\begin{abstract}
    Este é o resumo do trabalho, onde serão descritos de forma sucinta os principais pontos abordados.
\end{abstract}

\tableofcontents
\newpage

\section{Introdução e Motivação}
Contextualizar o problema e enfatizar o porquê é interessante estudar o tema?
Apresente uma visão geral do problema e explique a relevância do estudo.

\section{Conceitos Fundamentais}
Tudo que o leitor precisa saber antes para que possa entender seu trabalho.
Descreva os conceitos teóricos, definições e informações preliminares necessárias para compreender o trabalho.

\section{Trabalhos Relacionados}
Quais são os trabalhos na literatura que tentam resolver o mesmo problema?
Apresente uma revisão bibliográfica dos principais trabalhos e estudos relacionados ao tema do seu projeto.

\section{Objetivo}
De forma direta e sucinta, um parágrafo que resuma o que será feito neste trabalho.
Descreva claramente o objetivo principal do trabalho, destacando o que será alcançado.

\section{Metodologia Experimental}
Quais serão os passos e técnicas/biblioteca/tecnologias em geral que serão utilizadas para que seu projeto se concretize?
Detalhe a abordagem metodológica, os passos experimentais, as ferramentas e tecnologias que serão empregadas no desenvolvimento do projeto.

\section{O que será entregue no final?}
Esta parte é a mais importante, pois será a sua promessa de projeto e portanto, ela quem guiará sua nota final.
Explique quais serão os resultados finais, entregáveis ou produtos do seu trabalho, e como eles serão apresentados.

\section{Referências Bibliográficas}
Cada parágrafo deve conter a referência de onde tirou a ideia/definição/comentário.
Utilize o estilo ABNT para citar as fontes bibliográficas usadas no trabalho.
\begin{itemize}
    \item AUTOR, Nome. \textit{Título do livro}. Edição. Local de publicação: Editora, Ano.
    \item AUTOR, Nome. \textit{Título do artigo}. Nome do Periódico, Local de publicação, v. Volume, n. Número, p. Página inicial-final, Ano.
\end{itemize}

\end{document}
