\documentclass[a4paper,12pt]{article}
\usepackage[utf8]{inputenc}
\usepackage[T1]{fontenc}
\usepackage[brazilian]{babel}
\usepackage{indentfirst}
\usepackage{abntex2cite}
%\usepackage{hyperref}
\usepackage{graphicx}
\usepackage{amsmath}
\usepackage{setspace}
\usepackage{times}

\title{Predição do Estado de uma Smart Grid}
\author{João Vinicius Farah Colombini 159501\\
Victor Jorge Carvalho Chaves 156740}
\date{11/07/2024}

\begin{document}

\maketitle

\tableofcontents
\newpage

\section{Introdução e Motivação}

Redes elétricas são responsáveis por realizar a geração,
transmissão e distribuição de energia em um território
e são fundamentais para o funcionamento da sociedade.

E conforme o passar dos anos, com o crescimento da sociedade,
há o aumento no consumo de energia elétrica.
Além disso, com as questões climáticas em jogo e a busca por mais fontes de energia limpa,
há a entrada de novos elementos nas redes elétricas, como painéis solares, aerogeradores, etc. Que aumentam a complexidade das redes.

E por fim, ocorreu vários casos no mundo de blackouts, que foram causados por mal funcionamentos da rede, ataques cibernéticos, falta de manutenção, etc.

E com crescimento das redes elétricas para atender a situação do mundo, emergiu o conceito de Smart Grid (Rede Elétrica Inteligente), redes elétricas que implementam múltiplas tecnologias para lidar com os desafios citados acima.

E dentre umas das tecnologias aplicadas em Smart Grids, é a inteligencia artificial, que pode resolver desafios de forecasting, detecção de ataques, e problemas de otimização.

\section{Conceitos Fundamentais}
\subsection{Smart Grid}
Sistema de energia elétrica que se utiliza da tecnologia da informação para fazer com que o sistema seja mais eficiente (econômica e energeticamente), confiável e sustentável.

A definição de redes elétricas inteligentes ainda não está completamente consolidada, mas nesse sistema devem constar os seguintes atributos

\begin{itemize}
    \item Sistemas de transmissão e distribuição transparentes e controláveis;
    \item Fontes de energia renovável, geração distribuída e armazenamento de energia nos dois lados do medidor;
    \item Capacidade para resposta à demanda e controle de demanda.
\end{itemize}

\subsection{PyPSA: Python for Power System Analysis}
PyPSA é uma biblioteca de código aberto para simular e otimizar sistemas modernos de energia e energia que incluem recursos como geradores convencionais com compromisso de unidade, geração eólica e solar variável, unidades de armazenamento, acoplamento a outros setores de energia e redes mistas de corrente alternada e contínua.


\section{Trabalhos Relacionados}
\subsection{Harmonized and Open Energy Dataset for Modeling a Highly Renewable Brazilian Power System}

Nesse trabalho é desenvolvido um conjunto de dados para análise de cenários com modelos como o PyPSA.
Esse conjunto inclui dados de séries temporais, dados geoespaciais e dados tabulares sobre usinas e demandas de energia.
Isso facilita estudos adicionais focados na descarbonização do sistema energético brasileiro, mas pode ser auxiliar para outros estudo também.

\subsection{Machine Learning Approaches To Predict The Stability of Smart Grid}

Este estudo propõe um modelo de aprendizado de máquina para identificar a estabilidade da rede inteligente de forma mais eficiente.

\subsection{A multi-scale time-series dataset with benchmark for machine learning in decarbonized energy grids}

É um banco de dados gerado a partir de dados reais, e utiliza de séries temporais para isso. O processo pode ser descrito em três partes, a primeira é a coleta e geração de dados de carga e energia renovável, a segunda é Energia, voltagem e geração de dados e portanto a ultima é fundamentada em comparações de modelos de machine learning para atividades chave.

\section{Objetivo}

Este trabalho propões em treinar diversos modelos de inteligência artificial para conseguir chegar em um modelo ótimo que deverá ser capaz de categorizar e prever o estado de uma rede elétrica com o objetivo final de otimização energética.

\section{Metodologia Experimental}

\subsection{Tecnologias e Bibliotecas}
\begin{itemize}
    \item Linguagem de Programação: Python
    \item Biblioteca de Aprendizado de Máquina de IA: Scikit-learn, PyTorch
    \item Simulador de uma Rede Elétrica: PyPSA
\end{itemize}

\subsection{Etapas do Desenvolvimento}
\begin{enumerate}
    \item Definição e Criação de um Dataset usando PyPSA
    \item Pré Processamento dos Dados e Geração de Dados
    \item Treinamento e Teste de Modelos de IA de Regressão \begin{enumerate}
              \item Random Forests para regressão;
              \item Regressão linear;
              \item Árvores de decisão para regressão.
          \end{enumerate}
    \item Treinamento e Teste de Modelos de IA de Classificação \begin{enumerate}
              \item Arvore de Decisão;
              \item K-nearest neighbors;
              \item Árvores de decisão para regressão.
          \end{enumerate}
    \item Comparação e escolha do modelo a ser usado em cada caso.
    \item Buscar a possibilidade de fine tuning dos modelos achados para aplicar em outros desafios de mesmo domínio.
\end{enumerate}

\section{Entregáveis do Projeto}

O projeto visa desenvolver um modelo de inteligência artificial treinado com dados de redes elétricas, capaz de categorizar (classificação, dados discretos) e prever (regressão, dados continuos) o estado dessas redes com alta precisão.

Além disso, será feito a tentativa de adaptar o modelo para enfrentar outros desafios na área de Smart Grid, promovendo melhorias na eficiência, segurança e gestão dessas redes inteligentes.

\section{Referências Bibliográficas}
\begin{itemize}
    \item Deng, Y., Cao, KK., Hu, W. et al. Harmonized and Open Energy Dataset for Modeling a Highly Renewable Brazilian Power System. Sci Data 10, 103 (2023). https://doi.org/10.1038/s41597-023-01992-9
    \item T. Brown, J. Hörsch, D. Schlachtberger, PyPSA: Python for Power System Analysis, 2018, Journal of Open Research Software, 6(1), arXiv:1707.09913, DOI:10.5334/jors.188
    \item SAP Insights. "The Smart Grid: How AI is Powering Today’s Energy Technologies." Disponível em: SAP Insights. Acesso em: 11 jul. 2024.
    \item Satu, Md \& Khan, Md. Imran. (2024). Machine Learning Approaches To Predict The Stability of Smart Grid. 10.21203/rs.3.rs-3866218/v1.
    \item Y. Deng, "PyPSA-Brazil: A Free and Open Model of the Brazilian Electrical System," in Energy Proceedings, 2021.
    \item Zheng, X., Xu, N., Trinh, L. et al. A multi-scale time-series dataset with benchmark for machine learning in decarbonized energy grids. Sci Data 9, 359 (2022). https://doi.org/10.1038/s41597-022-01455-7
\end{itemize}
\end{document}
